\section*{Justificativa}
    Em teoria da computabilidade é apresenta o formalismo matemáticos
e os problemas algorítmicos solúveis, sendo definido problemas como
objetos matemáticos que têm interesse pŕoprios. Dessa forma, para a
resolução deve-se pensar me quais são os daddos, quais são os possíveis
resultados e se a resolução satisfaça, ou uma declaração de quais propriedades
a resposta ou solução deve satisfazer\cite{leal_2002_uma}.

    Assim sendo, eles podem ser classificados conforme demonstrado no livro Teoria
Computacional em Grafos em decisão, localização e otimização.
\begin{citacao}
Os problemas algorítmicos de otimização são aqueles que consistem em encontrar uma solução maxima
ou minima, de decisão são aqueles que precisão escolhar em determinada parte do problema entre sim e não
e os problemas de localização são aqueles que consistem em encontrar uma extrutura K que
atenda a determinada condição\citeauthor{szwarcfiter_2018_teoria}.
\end{citacao}
    
    Sendo alguns problemas de otimização, decisão e procura catalogados pelo seu crescimento
polinomiais, eles acabam sendo alocados no sub grupo NP-Completos e NP-Difícil, que estão contidos 
no grupos de algoritmos que são não deterministicos de tempo polinomial\cite{leal_2002_uma}.

    Por consequência, para encontrar um algoritmo que resolva com eficiência, é
um desafio ao pesquisadores. Contudo, com a realização da aplicação de computação por DNA
para resolução do problema do caminho hamiltoniano\cite{adleman_1994_molecular}, sendo possível
encontrar uma possivel solução para o problema NP-Completo, todavia foi usado uma semana
para a realização dos procedimos.

    Mesmo com a demora para realizar o algoritmo, segundo a pesquisa caso realizado a 
execução usando o DNA, levaria meses para descobrir se o grafo têm caminha hamiltoniano,
em comparação com os computadores eletrônicos no pior caso, a qual foi demonstrado as vantagens
conforme demonstado na Tese do \citeauthor{isaiafilho_2004_uma} (paralelismo, quantidade
de informação em um pequeno espaço). Sendo assim este projeto pretende investigar o uso de computação
por DNA ou computação Natural para a resolução de problemas NP-Completos.

