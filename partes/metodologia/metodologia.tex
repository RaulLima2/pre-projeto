\section*{Proposta Metodológica Preliminar}
    Para a realização deste estudo, será dividido em duas partes: a primeira parte, 
levantamento dos artigos publicados, e a segunda parte, que será a analise dos artigos.
Dessa forma, sera aplicado:
    \begin{itemize}
        \item{Levantamento de artigos:} serão pesquisados sobre Computaçaõ por DNA, NP-completos,Computação Natural
            segundo operações criteriso de pesquisa, como:
            \begin{itemize}
                \item{Palavra Chaves:} \textit{DNA Computing}, \textit{NP-Problem}, Computação Natural
                \item{Data de publicação:} 1994 $-$ 2022;
                \item{Autores:} Têm proeficiencia com o tema;
                \item{Índice de citações:} Número de citações por outros autores;
            \end{itemize}
            Sendo utilizado as plataformas Google Scholar, PubMed, Arvix, SpringerLink,
            ACM, IEEE, CiteSeer, Biblioteca Digital da Sociedade Brasileira de Computação,
            Scielo, Jstor, Journal Computer Biology e Repositorios das Universidades Federais e Estaduais.
        \item{Analise de artigos:} será usado para analisar os artigos publicados,a revisão sistemática, sendo elaborado uma ficha contendo: tema, autores, palavras chaves, metodologia, objetivo da
        pesquisa, objeto pesquisado, conclusção, importancia e problema encontrados.
    \end{itemize}