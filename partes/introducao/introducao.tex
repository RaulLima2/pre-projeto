\section*{Introdução}

    Seja o DNA uma forma de armazenar informações genéticas complexas, a 
qual foi proposto inicialmente pelos Gregos Antigos, que acreditavam na 
existencia da passagem de aspectos para proximas gerações. Depois deles 
o ciêntista Mendel acreditava que os parentes podiam contribuir com o
material genetico de um outro indivíduo, contudo com a descoberta 
dos chromosomes, descobriu que este continha proteina e o ácido dex
soxirribonucleico\footnote{DNA},
porém somente nos anos 50 foi aceita a ideia que o DNA contém o código genético,
sendo estudando sua estrutura para decifrar\cite{gibbons_1997} posteriomente.

    Dessa forma, ele é constituido por cadeias de polímeros que contêm 
cadeias de nucleotidos, que são compostos por um número de bases 
nitrogenadas e segue uma extrutura de ligação\cite{gibbons_1997}.

    Por consequência, essa extrutura possui alguamas propriedades de 
operações básicas: sintese, desnaturar, ligar e separar filamentos. 
Também existe ténicas de manipulação como Electroforese em gel e a
utilização do PCR\footnote{Reação em cadeia da polimerase}\@.

    A partir disso, alguns pesquisadores pensaram que era possível 
utilizar as cadeias de DNA para criar um micro computador que apresentasse
uma performance melhor que a de um computador tradizional usando o DNA\@\cite{conrad_1985_on}.
    
    Por conseguinte, uma primeira proposta de solução para o problema, foi de elaboração
de um microcomputador que porderia ser manipulado por uma molécula, a qual foi 
proposta por Richar Feyman, em  sua palestra ''\textit{There's Plenty of Room at the Bottom}'',
a qual relatou  que as células além de serem em menor escala, podem armazenas informações em
seu interior\cite{feynman_1992_theres}. 

    Após isso, Charles Bennett em \citeyear{bennett_1982_the}, propros o conceito baseado
no movimento Browniano, sendo este definido pela reação, efetivação do estado de transição
e suspensão de fluido, sendo desenvolvido o conceito de Máquina de Turing Browniana, que consistia
em utilizar moléculas de RNA para realizar operações.

    Igualmente, nos trabalhos realizados por Conrad e Liberman\citeyear{conrad_1982_molecular} introduziram a relação entre
a física e a computação de processos, sendo descritos por processamento de palavras, cinemática
informacional de processamentos; mudança de conformidade macromoléculas; distribuição
de controle de menbrana e computação disfuncional.

    Em outra pesquisa realizada pel Conrad, publicada em \citeyear{conrad_1985_on}, o artigo 
apresenta como poderia ser utilizadas as moléculas de forma que poderia substituir as moleculas
também  apresentou sobre a questão de eficiencia versus custo, na elaboraçaõ de um sistema que 
utilizar circuitos contra um que utiliza o DNA\@.

    Contudo, somente em 1994\cite{isaiafilho_2004_uma}, foi possivel testar a  possibilidade
de um computador por DNA, por meio de um experimento realizado pelo pesquisador \citeauthor{adleman_1994_molecular},
a qual resolveu um problema NP-Completo, que seria um Problema de tempo polinomial chamado
Problema do Caminho Hamiltoniano que consistia em encontrar um caminho hamiltoniano
em um grafo, para resolução foi usado a força-bruta. Por fim, dando inicio a era da
Computação por DNA\@.